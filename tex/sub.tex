
\documentclass{article}
\title{Sub Operator}
\date{10/1/2023}
\author{Rylan W. Yancey}
\begin{document}
    \maketitle
    \section*{Definition \& Gradient}
    The Addition Operator is defined as $f(u,v) = u - v$. To find the gradient w.r.t u and v, 
    we will use the subtraction rule, which is defined as: 

    $$\frac{\delta}{\delta{x}}(u - v) = \frac{\delta{u}}{\delta{x}} - \frac{\delta{v}}{\delta{x}}$$

    To find the gradient w.r.t u, we will set u as our x and treat v as a constant, 
    which gives us the following:

    $$\frac{\delta}{\delta{u}}(u - v) = \frac{\delta{u}}{\delta{u}} - \frac{\delta{v}}{\delta{u}} = 1$$

    To find the gradient w.r.t v, we will set v as our x and treat u as a constant, 
    which gives us the following:

    $$\frac{\delta}{\delta{v}}(u - v) = \frac{\delta{u}}{\delta{v}} - \frac{\delta{v}}{\delta{v}} = -1$$

    Therefore, we can say that the gradient w.r.t u is 1, and the gradient w.r.t v is -1. 

\end{document}