
\documentclass{article}
\usepackage{amsmath}
\title{Sigmoid Operator}
\author{Rylan W. Yancey}
\date{10/1/2023}
\begin{document}
    \maketitle
    \section*{Definition \& Gradient}
        The Sigmoid function is defined as $\sigma(x) = \frac{1}{1+e^{-x}}$ To find the gradient, we will apply the chain rule and simplify.
        \begin{align}
            \dfrac{d}{dx} \sigma(x) &= \dfrac{d}{dx} \left[ \dfrac{1}{1 + e^{-x}} \right]\\
            &= \dfrac{d}{dx} \left( 1 + \mathrm{e}^{-x} \right)^{-1} \\
            &= -(1 + e^{-x})^{-2}(-e^{-x}) \\
            &= \dfrac{e^{-x}}{\left(1 + e^{-x}\right)^2} \\
            &= \dfrac{1}{1 + e^{-x}\ } \cdot \dfrac{e^{-x}}{1 + e^{-x}}  \\
            &= \dfrac{1}{1 + e^{-x}\ } \cdot \dfrac{(1 + e^{-x}) - 1}{1 + e^{-x}}  \\
            &= \dfrac{1}{1 + e^{-x}\ } \cdot \left( \dfrac{1 + e^{-x}}{1 + e^{-x}} - \dfrac{1}{1 + e^{-x}} \right) \\
            &= \dfrac{1}{1 + e^{-x}\ } \cdot \left( 1 - \dfrac{1}{1 + e^{-x}} \right) \\
            &= \sigma(x) \cdot (1 - \sigma(x))
        \end{align}
    Therefore, we say that the gradient of the sigmoid function with respect to x is $\sigma(x) \cdot (1-\sigma(x))$.

    \subsection*{Contributions}
    Thanks to Michael Percy for the detailed steps.
\end{document}